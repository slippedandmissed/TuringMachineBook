\section{Creating a Decoder}

Now that we've got a working encoder which lets us transform our ruleset into machine code, we can start creating a decoder which allows us to turn machine code back into a ruleset.

One method which we will of course need is \cpp{vlq::readFromStream} which we delcared in a previous section but have not yet implemented. This method is supposed to read a VLQ-encoded value from an input stream and return it as a \cpp{state_t}.

We will define this method in \path{src/vlq.cc}.

\begin{file}{src/vlq.cc}{c++}{36}
state_t vlq::readFromStream(std::istream &is)
{
    state_t value = 0;
    bool moreOctets;
    do
    {
        unsigned char octet = is.get();
        moreOctets = (octet & 0x80) >> 7;
        value <<= 7;
        value |= octet & 0x7F;
    } while (moreOctets);
    return value;
}
\end{file}

It starts off with a value of zero, and then reads an octet from the input stream. It appends the least significant 7 bits of this octet to the running total, and repeats until it finds an octet whose most significant bit is a 0, indicating that the current octet is the last one for this value.

We will also need a data structure to contain our decoded Turing machine program once we decode it. For the moment, this could simply be a \cpp{std::vector<Rule>} object, but when we add program metadata to the machine code format, this structure will need to hold some additional information. For now, we will describe a simple wrapper struct in a new header, \path{src/program.hh}.

\begin{file}{src/program.hh}{c++}{1}
#pragma once
#include <vector>
#include "rule.hh"

struct Program
{
    std::vector<Rule> rules;
};
\end{file}

Using these tools, we will be able to implement the decoder. We will do so in a new file, \path{src/decoder.cc}, with a corresponding header file \path{src/decoder.hh}.

Like for the encoder, the decoder logic will be contained within a \cpp{Decoder} class. However, since we no longer need to worry (for the moment) about translating between string states and integer states, this class will be much simpler than \cpp{Encoder}. It will have a single method, \cpp{decode}, which takes in an input stream and returns a pointer to one of our \cpp{Program} objects.

\begin{file}{src/decoder.hh}{c++}{1}
#pragma once
#include <memory>
#include <istream>
#include "program.hh"

class Decoder
{
public:
    std::unique_ptr<Program> decode(std::istream &);
};
\end{file}

The reason this method returns a \cpp{Program} pointer rather than simply a \cpp{Program} is so that we can return a \cpp{nullptr} if the decoding fails.

We can define this method in \path{decoder.cc}. The first thing we want to do is check that the 6 magic bytes \cpp{"TURING"} appear as the first 6 bytes of the stream.

\begin{file}{src/decoder.cc}{c++}{1}
#include "decoder.hh"
#include "vlq.hh"

std::unique_ptr<Program> Decoder::decode(std::istream &is)
{
    std::string magicBytes(6, '\0');
    is.readsome(&magicBytes[0], 6);

    if (magicBytes != "TURING")
    {
        return nullptr;
    }
\end{file}

Next we want to initialise a \cpp{Program} object to contain our decoded data. We can use \cpp{vlq::readFromStream} to read the value encoding the number of rules in the ruleset. We'll use this value to set the size of the program's \cpp{rules} vector.

\begin{file}{src/decoder.cc}{c++}{13}
    Program program;

    state_t numRules = vlq::readFromStream(is);
    program.rules.resize(numRules);
\end{file}

Now, we can iterate over each rule and read its data from the input stream, storing it in the appropriate \cpp{Rule} object in the \cpp{rules} vector.

\begin{file}{src/decoder.cc}{c++}{18}
    for (state_t i = 0; i < numRules; i++)
    {
        program.rules[i].startState = vlq::readFromStream(is);

        std::getline(is, program.rules[i].matchSymbol, '\0');
        std::getline(is, program.rules[i].replaceSymbol, '\0');

        program.rules[i].direction = static_cast<Direction>(is.get());

        program.rules[i].endState = vlq::readFromStream(is);
    }
\end{file}

Finally, we return a pointer to the \cpp{Program} object we've created.

\begin{file}{src/decoder.cc}{c++}{30}
    return std::make_unique<Program>(program);
}
\end{file}

Note that the only error-checking that we are doing is for the magic bytes. We are not checking that we have actually managed to decode the expected number of rules, whether \cpp{vlq::readFromStream} fails due to reaching the end of the stream without finding a terminating octet, or indeed any other way in which this decoder could fail with bad input.

However, this is going to be sufficient for our purposes. The magic byte check will tell us whether we have, for example, passed the decoder a file which isn't even a Turing machine code file. Otherwise, we will assume that any machine code file that we pass to the decoder will be correctly formed.

If you feel so inclined, I encourage you to modify this decoder to be more informative in the case of an error, perhaps even displaying some human-readable error messages.

To be able to compile the decoder, we must add a target to our Makefile.

\begin{file}{Makefile}{make}{15}
decoder: dist-dir
	$(CC) $(CFLAGS) -o dist/decoder.o -c src/decoder.cc
\end{file}