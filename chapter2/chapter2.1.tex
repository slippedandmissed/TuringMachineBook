\section{Definition}
A Turing machine is a theoretical machine first described by Alan Turing in 1936. This machine consists of a tape (a long line of adjacent cells) extending infinitely far in both directions. Each cell contains a symbol. The machine has a read/write head positioned over one of the cells, which is capable of determining which symbol is written in that cell, and is also able to replace it with a different symbol. The machine also has some internal state which can change over time.

This machine follows a particular finite set of so-called \textit{transition rules}. Each rule has the format:

\begin{quote}
    If the machine is in state \statename{X}, and the symbol under the read/write head is symbol \symb{A}, then replace it with symbol \symb{B}, move the read/write head exactly one cell to the left/right, and then set the state to state \statename{Y}.
\end{quote}

which can be expressed more concisely as:

\begin{quote}
    $\Delta(X,A) = (B,\shortleftarrow,Y)$
\end{quote}
or
\begin{quote}
    $\Delta(X,A) = (B,\shortrightarrow,Y)$
\end{quote}
depending on whether the read/write head should move to the left or right, respectively.

Consequently, this machine can carry out computations. The set of transition rules are like a program, and the initial symbols on the tape are like the input to that program. The machine starts in some pre-defined state, repeatedly applies whichever rule is appropriate, and once it reaches a situation in which none of its rules apply, it terminates leaving whichever symbols remain on the tape to serve as the output.

The incredible thing about this hypothetical machine is that it can actually compute any computable algorithm. If an algorithm can run on a modern computer, it can be simulated with a finite set of Turing machine transition rules.

It is with this theorem that Turing defined the set of all computable programs as the set of programs which can be executed on a Turing machine. As such, if some machine (or say, programming language) can be used to simulate a Turing machine, it then follows that it can also compute any computable algorithm.

This is the main reason why we can be confident that we can compile high-level code into Turing machine transition rules --- any computable algorithm can be run on a Turing machine.

In order to clarify our goals, we can define what we mean by a Turing machine program mathematically.

\begin{align*}
    &S:=\{\textrm{possible states}\} \textrm{ such that }|S|\textrm{ is finite} \\
    &A:=\{\textrm{possible symbols}\} \textrm{ such that }|A|\textrm{ is finite} \\
    &D:=\{\shortleftarrow, \shortrightarrow\}\\
    &s_0 \in S\\
    \\
    &\Delta: (S\times A) \pfun (A \times D \times S)
\end{align*}

Here, $S$ is the finite set of possible states that the Turing machine can be in. $A$ (short for alphabet) is the finite set of symbols which can be written on a cell in the tape. The Turing machine initially starts in state $s_0$.

$\Delta$, the set of rules, is a partial function between the sets shown above. The cell under the read/write head, the position of the read/write head, and the state of the machine are all updated repeatedly according to this function, until no further rule applies.

Note that we have not said anything about what the individual elements of $S$ and $A$ might be. In fact, these elements will depend on the program we are running (i.e., some programs might require more states than others, or a different alphabet. Regardless, both of these sets will always be finite).

We will have a lot of flexibility when implementing the Turing machine simulator. For example, our states could be represented by natural numbers (state 0, state 1, etc.) or by strings (\cpp{"start"}, \cpp{"ready"}, etc.). We will switch between these two representations when it becomes helpful to do so.

Similarly, we haven't specified what type of symbols we are using. When we actually come to implement the simulator, we will use strings to represent the symbols (most will only be a single character, but it will become very helpful to allow for multi-character symbols).

I should note that some may argue that $A$ should only have exactly two elements (e.g., 0 and 1). This restriction doesn't really impact the computing power and would only serve to make the journey ahead much less enjoyable, so we will choose to ignore it.
