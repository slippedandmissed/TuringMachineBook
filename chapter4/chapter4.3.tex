\section{Testing the Simulator}
We will write some simple tests for our simulator using the program from page \pageref{program:simpleTest}, for which the machine code is:
\begin{stdout}
54 55 52 49 4E 47 09 00 31 00 30 00 01 01 00 30 00 31 00 00 00 00 00 00 01 02 02 31 00 00 01 02 01 30 00 30 00 01 01 01 31 00 31 00 01 01 01 41 00 41 00 01 01 01 00 41 00 00 00 00 41 00 41 00 00 00
\end{stdout}
As a reminder, this program takes a multi-cell input in binary, and outputs a string of \cpp{"A"}s which is that many characters long.

We will put these tests in \path{src/tests/simulator-tests.cc}. The first test will be with an empty input. We would expect the simulator to finish in state \statename{2} (\statename{done}) and the tape should be empty.

\begin{file}{src/tests/simulator-tests.cc}{c++}{1}
#include <sstream>
#include "tests.hh"
#include "../decoder.hh"
#include "../simulator.hh"

static bool emptyStringOfAsTest(void)
{
    std::stringstream is(std::string(
        "\x54\x55\x52\x49\x4E\x47\x09\x00\x31\x00\x30\x00\x01\x01\x00\x30"
        "\x00\x31\x00\x00\x00\x00\x00\x00\x01\x02\x02\x31\x00\x00\x01\x02"
        "\x01\x30\x00\x30\x00\x01\x01\x01\x31\x00\x31\x00\x01\x01\x01\x41"
        "\x00\x41\x00\x01\x01\x01\x00\x41\x00\x00\x00\x00\x41\x00\x41\x00"
        "\x00\x00",
        66));
    Decoder decoder;
    auto program = decoder.decode(is);

    std::vector<std::string> initialSymbols;
    Tape tape(initialSymbols);

    Simulator sim(std::move(program), tape);

    while (sim.step())
    {
    }

    std::vector<std::string> expected({""});

    return sim.getState() == 2 && tape.getAllSymbols() == expected;
}
\end{file}

As you can see, we start by loading the machine code into a string stream, and then decode it using the decoder to get a pointer to the program. Then, we create an empty tape (as the input is empty for this test), and then use these to construct the simulator.

We then run the \cpp{sim.step} method until it returns \cpp{false}, which indicates that the program has finished running. Then we check that the ending state is indeed \statename{2} and that the tape ends up with a single empty cell.

The next test will be very similar except it will test what happens when the input is a single cell containing the symbol \cpp{"0"}. The expected output is the same.

\begin{file}{src/tests/simulator-tests.cc}{c++}{32}
static bool stringOf0AsTest(void)
{
    std::stringstream is(std::string(
        "\x54\x55\x52\x49\x4E\x47\x09\x00\x31\x00\x30\x00\x01\x01\x00\x30"
        "\x00\x31\x00\x00\x00\x00\x00\x00\x01\x02\x02\x31\x00\x00\x01\x02"
        "\x01\x30\x00\x30\x00\x01\x01\x01\x31\x00\x31\x00\x01\x01\x01\x41"
        "\x00\x41\x00\x01\x01\x01\x00\x41\x00\x00\x00\x00\x41\x00\x41\x00"
        "\x00\x00",
        66));
    Decoder decoder;
    auto program = decoder.decode(is);

    std::vector<std::string> initialSymbols({"0"});
    Tape tape(initialSymbols);

    Simulator sim(std::move(program), tape);

    while (sim.step())
    {
    }

    std::vector<std::string> expected({""});

    return sim.getState() == 2 && tape.getAllSymbols() == expected;
}
\end{file}

Likewise, the next test will be for the input \cpp{"1"}. Here we expect the tape to end up with a single cell containing the symbol \cpp{"A"}.

\begin{file}{src/tests/simulator-tests.cc}{c++}{58}
static bool stringOf1ATest(void)
{
    std::stringstream is(std::string(
        "\x54\x55\x52\x49\x4E\x47\x09\x00\x31\x00\x30\x00\x01\x01\x00\x30"
        "\x00\x31\x00\x00\x00\x00\x00\x00\x01\x02\x02\x31\x00\x00\x01\x02"
        "\x01\x30\x00\x30\x00\x01\x01\x01\x31\x00\x31\x00\x01\x01\x01\x41"
        "\x00\x41\x00\x01\x01\x01\x00\x41\x00\x00\x00\x00\x41\x00\x41\x00"
        "\x00\x00",
        66));
    Decoder decoder;
    auto program = decoder.decode(is);

    std::vector<std::string> initialSymbols({"1"});
    Tape tape(initialSymbols);

    Simulator sim(std::move(program), tape);

    while (sim.step())
    {
    }

    std::vector<std::string> expected({"A"});

    return sim.getState() == 2 && tape.getAllSymbols() == expected;
}
\end{file}

And finally, we will test the input 5, which this program expects in binary, with one cell per digit (\cpp{"1"}, \cpp{"0"}, \cpp{"1"}). The expected output is 5 cells, each containing the symbol \cpp{"A"}.

\begin{file}{src/tests/simulator-tests.cc}{c++}{84}
static bool stringOf5AsTest(void)
{
    std::stringstream is(std::string(
        "\x54\x55\x52\x49\x4E\x47\x09\x00\x31\x00\x30\x00\x01\x01\x00\x30"
        "\x00\x31\x00\x00\x00\x00\x00\x00\x01\x02\x02\x31\x00\x00\x01\x02"
        "\x01\x30\x00\x30\x00\x01\x01\x01\x31\x00\x31\x00\x01\x01\x01\x41"
        "\x00\x41\x00\x01\x01\x01\x00\x41\x00\x00\x00\x00\x41\x00\x41\x00"
        "\x00\x00",
        66));
    Decoder decoder;
    auto program = decoder.decode(is);

    std::vector<std::string> initialSymbols({"1", "0", "1"});
    Tape tape(initialSymbols);

    Simulator sim(std::move(program), tape);

    while (sim.step())
    {
    }

    std::vector<std::string> expected({"A", "A", "A", "A", "A"});

    return sim.getState() == 2 && tape.getAllSymbols() == expected;
}
\end{file}

To tie these tests together, we will implement this component's \cpp{allTests} method.

\begin{file}{src/tests/simulator-tests.cc}{c++}{110}
void tests::simulator::allTests(tests::Record &record)
{
    RUN_TEST(emptyStringOfAsTest, record);
    RUN_TEST(stringOf0AsTest, record);
    RUN_TEST(stringOf1ATest, record);
    RUN_TEST(stringOf5AsTest, record);

    return;
}
\end{file}

To make sure that these tests are run, we must add this component to the \cpp{FOR_ALL_COMPONENTS} macro in \path{src/tests/tests.hh}.

\begin{file}{src/tests/tests.hh}{c++}{8}
#define FOR_ALL_COMPONENTS \
    X(encoder)             \
    X(decoder)             \
    X(vlq)                 \
    X(simulator)
\end{file}

Finally, we must add our new source files and dependency targets to the \code{tests} target in the Makefile.

\begin{file}{Makefile}{make}{4}
tests: vlq encoder decoder tape simulator
	$(CC) $(FLAGS) -o dist/tests dist/encoder.o \
	                             dist/decoder.o \
	                             dist/vlq.o \
	                             dist/tape.o \
	                             dist/simulator.o \
	                             src/tests/vlq-tests.cc \
	                             src/tests/encoder-tests.cc \
	                             src/tests/decoder-tests.cc \
	                             src/tests/simulator-tests.cc \
	                             src/tests/tests.cc
\end{file}

Now, when we run our test suite, all of our tests should pass, indicating that our simulator is working as expected.
\begin{stdout}
Running encoder tests:
  simpleTest: ✅
PASSED: 1/1 ✅

Running decoder tests:
  simpleTest: ✅
PASSED: 1/1 ✅

Running vlq tests:
  is0EncodedCorrectly: ✅
  is50EncodedCorrectly: ✅
  is123456789EncodedCorrectly: ✅
  is0DecodedCorrectly: ✅
  is50DecodedCorrectly: ✅
  is123456789DecodedCorrectly: ✅
  is123456789DecodedCorrectlyWithMoreBytes: ✅
PASSED: 7/7 ✅

Running simulator tests:
  emptyStringOfAsTest: ✅
  stringOf0AsTest: ✅
  stringOf1ATest: ✅
  stringOf5AsTest: ✅
PASSED: 4/4 ✅

OVERALL PASSED: 13/13 ✅
\end{stdout}